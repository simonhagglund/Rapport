\section{Didactics of Hard Science}
\todo[inline]{This section is still very much work in progress.}

\iffalse
\textbf{Point of this section}: Present relevant didactic theory.
\textbf{end goal}: Validate project
\fi

During their journeys towards degrees, students will likely meet as many didactic methodologies as they take courses (and possibly more). These may or may not be built on explicitly built on didactic theory, but they can likely be described from a constructivistic viewpoint. 
% Instead of: 
% When university students embark on the journey to a master degree, during each course they will encounter some didactic methodology that is most frequently at present day and age well described from a constructivistic standpoint.
Constructivism is a set of ideas describing how a person internalises knowledge. The basic idea is that a person constructs their knowledge, as contrasted with classical teacher-based education where a teacher gives a student knowledge.
Constructivism invariably details an epistemology (a model of what knowledge is) and gives some explanations about how to construct such knowledge. There are different versions of constructivism, and most new science in the field supports the old ideas. As it is an old theory with a wealth of supporting evidence\todo[color=other]{Is this true?}, there is a widespread acceptance of constructivism. However, the mode of learning differs between different constructive theories. \cite[p.51-53]{imsen_2005}
%In any didactic context constructivism is somewhat of a buzzword, aiming to explain the process of learning according to some epistemology, what knowledge is, new science in the field for the most part confirms or support the old ideas, and as such there is widespread acceptance of constructivism, but the flavour of constructivism may differ\cite[p.51-53]{imsen_2005}. 


The epistemological ideology of constructivism differs substantially from any common view, since mostly everyone lives as realists---the chair is a chair because I can see it--- % maybe "because of its inherent quality of being chair"? Get's really weird when discussing epistemology...
constructivism must accept a contrary position where our senses does not have any inherent order or structure other then the description we give them by some social convention or alike. Knowledge does not depend on the actual phenomena in the world, but is rather a construct of the learner or knower \cite[p.49]{imsen_2005}. The epistemological perspective that knowledge is nothing else then constructed, dictates that the pedagogy and tools provided must engage the students or, as O'loughlin said: 
%% Detta citatet beskriver rätt bra vad vi vinner om vi går med på de epistomologiska förlusterna.
''The power of the image lies in the contrast between the passive, powerless learner in the traditional approach, and this 
image of an active, constructive knower, empowered to take charge of his or her own learning. [...] [C]onstructivism makes a strong appeal to our commonsense understanding of how learning ought to be.'' \cite[p.792]{oloughlin_2007}, this ties into the anecdotal ability of some people that are able to recite glosses without understanding what the words mean, showcasing a shift in education.

%%This act of constructing is the corner stone of the pedagogy and the primary tool by which students learn, according to constructive didactics. % did I butcher the sentence? Is it still saying what it's supposed to?

Typically, the didactic systems used in Computer Science education is well depicted by sociocultural constructivism, where assignments are solved by a group and tasks are designed more to teach the students high- and lowlevel skills such as conceptualisation and how to correctly write programs rather than to widen knowledge of a particular way of doing things in a subject such as mathematics \cite{jenkins_2001}. 
Contrarily, the educational strategies typically employed in the hard sciences is better described by constant cycles of assimilation and accomodation. \todo[color=lessurgent]{Describe what assimilation and accomodation are?}


 %The CS programs didactic mythology is frequently akin to a sociocultural constructivism with assignments solved by a group and task designed not forthmost to widen knowlage of a framework but to infer in the students high- and low level skills \cite{jenkins_2001} i.e. conceptualisation of solutions to problems \& correctly write programs, whereas hard sciences corresponding environment is well described by constant cycle of assimilation and accommodation.


...
Teaching programming is entangled with much difficulty, ... but this is stuff that CS students learn with much ado. ect.% ?? 




\subsection{Motivation}

\textbf{Point of this section}: Present theory about motivation.
\textbf{end goal}: Claim that motivation is most important but abstract.
\newline

When a student learns programming, the student learns a practical skill, but it is no easy feat. It requires students to engage with coding whether there is credits to be made or not, \cite{jenkins_2001} to practise iteratively. The student must accept and get used to a style of teaching where the theoretical and practical intertwines in tasks that must be considered a good idea \cite{jenkins_2001}. From this easily follows that CS-students are used to a methodology of learning, which no small part consist of is writing code to internalise by exploration.

Most things in programming lacks a real life example, application of programming concept \cite{dunican_2002}.

\subsection{Two Dimensions}

\textbf{Point of this section}: Present theory about that the product works in both ways.
\textbf{end goal}: Claim some more justification then only made for CS students that know how to code.
\newline

Det finns två dimension här, dels att programmering är en enkel väg in i reglerteknik, men också kanske lite oväntat det omvända, att regler tekniken är en väg in i programmeringen. 

Läser jenkis som hävdar att en stor andel av de som går program som innehåller programmering eller där det utgör en stor del av batchelor eller master, egentligen vill undvika eller inte är motiverade att lära sig koda. \cite{jenkins_2001}

\subsection{Constructivism}

Knowledge is constructed by using the learners’ prior knowledge as foundation.

\begin{enumerate}
    \item Which knowledge needs to be connected?
    \item Which skills are needed to connect knowledge?
    \item What is an authentic task and how to specify it?
\end{enumerate}

According to constructivism, knowledge needs to be
related to each other and integrated to knowledge structures
that can be retrieved at any time in order to be useful for
problem solving. 

\begin{enumerate}
    \item Prerequisite skills, skills needed to enter the field.
    \item Specific skills, skills needed to perform key tasks in the field.
    \item Generic skills that are important for solving authentic tasks and evaluating correctness.
\end{enumerate}

\textbf{Point of this section}: Present theory about that math and sience mostly relies on Piaget Constructivism and CS on Vygotsky.
\textbf{end goal}: Claim that this project bridges the constructivism of Math \& Sience and CS, making it easier to for CS students.
\newline


There is two main branches of Constructivism..

"However, there exists a polarization between the Piaget’s
and Vygotsky’s paradigms"

\subsubsection{Zone of Proximal Development}
\textbf{Point of this section}: Present ZPD
\textbf{end goal}: Claim justification for what the product aims to do.
\newline

Is the learning zone where a student is capable of learning with the help of others.. (the product aimes to push student into this zone by relating stuff the student knows to what the student does not yet fully grasps \cite{vygotski_1978}.


\subsection{Instructional Scaffolding}
\textbf{Point of this section}: Provide didactic context. What it is. (ZPD and Scaffolding covers it)
\textbf{end goal}: Account for full teaching environment. 
\newline

Scaffolding is the support structure from teachers for the student.. 

This is what the product aims to be a part of.

"ZPD and scaffolding complement the need
for activation of prior knowledge to understand programming concepts. It is relevant in
rich learning environments which afford learners the ability to build a community with
their peers and teachers." from  \cite{wood_bruner_ross_1976}


