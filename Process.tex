\section{Process}\label{sec:process}
%Previously, some different ideas about how to present mathematical theory have been mentioned. 
%The learning material has been created according to the following principles. \todo{Något annat ord än principle?}


%Some proposed ideas to help communicate mathematics are to explicitly state the types included in expressions and to distinguish between syntax and semantics \cite{wells1995communicating}. Furthermore, using \gls{DSL}s to help understanding has also been proposed \cite{dslsofmath}\todo{Bättre med \cite{Ionescu_2016}?}.

% the first main idea it's built on is this, and here's an example from the learning material: 


%\todo{Do we want something more about \gls{didactic}s? This is a bit short and... intetsägande.}

%\todo[inline]{Do we want to mention that we wrote the learning material in \LaTeX{}?}% \todo[inline]{Syntax och semantik kan särskiljas med GADTs, see section \ref{sec:edsl}.}
% \todo[inline]{Koppla uppbyggnaden av läromaterialet till didaktiska teorin.}
% \todo{Skrev i latex, kanske}

The learning material is developed based on a product specification created to fit the aims of the project. 
%The theories presented in sections \ref{sec:theory} and \ref{sec:didactics} are
\begin{modtext}
The theory presented in section \ref{sec:theory} is
\end{modtext}
used to construct the learning material in an appropriate form.



\iffalse
The first step in our process was to set apart the first few weeks for study purposes. The team believed this would help minimise errors later in the process as well as improve flow. In addition, contact was established with teachers in order to pin down which parts of the course students struggled with the most. This would allow the team to narrow the scope of the project and ensure the most critical areas were adequately covered.

The next step was to create the DSLs themselves. To this end, the critical areas of study were split apart, and responsibility for each was distributed across the team. Each team member was given free reign to design their DSLs however they believed would offer the most effective explanation.

The third step was to start writing the text itself. Each team member was tasked with describing the DSLs they had created in accordance with the requirements described in Task(3.2), above. Though only a single person was responsible for each DSL and associated text, the team was heavily encouraged to employ pair programming to complete their tasks. \LaTeX was used as a repository for the text, under the pretext that it would aid in converting the text to a website at a later date, if time allowed for it.




In a perfect world, there would be opportunity to conduct a study on how the product facilitates learning. However, due to time constraints and the fact that there is no instance of the course in question concurrent to development. One proposed solution -- inspired by ``TSS with DSLs'' -- was to ask previous students of the course to read the material and fill in a survey. 
    

Finally, in order to evaluate the team's work and whether the goals are met, the material should be tested on a group of students. Unfortunately, there will be no concurrent courses held which can be used to test the learning material. One possible solution to this---inspired by ``TSS with DSLs''---is to ask previous students of the course to test the learning materials and fill in a survey. 
To make this evaluation more thorough the students could be split up into a test group and a control group who would spend an equal amount of effort on the same concepts. The former using our material and the latter using the material offered in the course, exclusively. However, it might not be possible to assemble enough participants to have the separate groups be large enough for this to be worth doing. 
\fi


\subsection{Product Specification}
%%In order to construct a learning material for control theory, the members of this project has to themselves understand the subject. Thus, a lot of focus has been put towards this aspect of the project. However, the central part of this project has to revolve around finding where domain-specific languages may be used in the domain and implement these. The details of this process is discussed in section \ref{sec:process}.  %% This is Process, move or remove

%\todo{- Lista ut hur det funkar DSL-mässigt 
% - Hur man beskriver DSLer för folk
% - Testa på studenter.} 
% Svar: Hur vi uttrycker oss, och vad vi menar. Syntax och semantik.
\iffalse
produkt - 
kontinuerligt rapportera processen
kontakt med examinator
be studenter som gått/går kursen att ge respons
skriva en "bok" i latex
ev göra om boken till en hemsida

hur ska vi testa det?
prata med folk somh har läst kursen innan 

rapport - 


Känns som vi bör diskutera metod mer på nästa möte.
\fi

\iffalse
Metod/Genomförande
Hur gruppen tänkt sig att genomföra arbetet Olika deluppgifter/delstudier kräver ofta separata metodavsnitt.

Metodbeskrivningen förankras vanligen i metodlitteratur. 

Detta är typiskt ett avsnitt som uppdateras under arbetets gång
\fi

The learning material should have certain properties. Its purpose is to work as a learning material for students of the course \gls{ERE103}. 
The learning material should not act as a replacement of the course book but instead be a supplementary material that offers a different approach to learning \gls{controltheory} and provide a perspective more familiar to computer science students, in order to ease the transition between the subjects.

The learning material should accomplish this by constructing \gls{DSL}s for different areas of \gls{controltheory}. Some, but not all, code should be presented and explained in the text. The programming language used should be \gls{Haskell}. \gls{GADTs} should be used. The text should not use very formal language but instead be light and use pictures to be more easily digestible for the reader. The text should have examples and exercises.

As it is a large and difficult course as established in section \ref{backgroundERE} it should not be expected that this learning material will be able to cover the entirety of the material covered in the course, or even the majority of it. As such the material should focus on the initial areas of the course, including some prerequisite knowledge. 






\subsection{Delimitations}\label{sec:delimitation}
Control Theory is a large subject. As such, trying to create a comprehensive guide to the entire subject would risk limiting the quality of each individual concept. Therefore, an early part of the project was to analyse which concepts of \gls{controltheory} to focus the team's efforts on. %nice


Using feedback from staff, records of prior exams, as well as the team's own experiences of the subject, subject areas were established and analysed. The analysis included, in a rough order of importance, which subjects students frequently struggled with, which concepts could be effectively described using \gls{DSL}s, and which other \gls{didactic} techniques could be employed.

Ultimately, five subject areas were chosen for further work:
\begin{itemize}
    \item Prerequisites
    \item Basic Control Theory
    \item Laplace Transform
    \item Transfer Functions
    \item Nyquist criterion of stability
\end{itemize}

Prerequisites and Basic Control Theory are both rundowns of multiple smaller subjects that should provide little issue but are nonetheless important to bring up. Laplace, Transfer Functions and Nyquist are larger and more difficult subjects that were deemed worthy of their own dedicated sections.

There was also a plan to include a section covering Bode diagrams. However, due to time constraints, it had to be cut.

\begin{newtext}
As previously discussed in section \ref{intro}, there is overlap between the content covered in the learning material and previous projects.
One such project, henceforth ``TSS with DSLs,'' used a similar approach to teach the course \course{Transforms, Signals and Systems}{Transformer, Signaler och System}{SSY080} and was an important inspiration for this project. ``TSS with DSLs'' created \gls{DSL}s for, among others, complex numbers, signals and LTI systems. %are all covered by a previous project (which will be called ``TSSwithDSLs''\cite{tssarbete};
As these subjects were to be included in the learning material, the choice whether to use the already developed \gls{DSL}s or to develop new ones had to be taken. 
New \gls{DSL}s were developed for all these domains, and the \gls{DSL} for complex numbers was used as a dependency for the \gls{DSL} for the Laplace transform. 
%however, many of the implementations did not suit the intended approaches in the learning material. 
Similarly, \gls{DSLsofMath} develops a \gls{DSL} for complex numbers, which is not used in the learning material. 
See section \ref{sec:discussion} for more information on the choice to make new \gls{DSL}s instead of using the previously developed ones.
%Similarly, complex numbers and the Laplace transform is described by ``DSLsofMath,'' \cite{dslsofmath} but the approach taken there was not what was envisioned in this project. Thus all of the \gls{DSL}s for these subjects were implemented from scratch, with one exception: the \gls{DSL} for complex numbers developed in a previous project \cite{tssarbete} was used alongside the one developed from scratch in the learning material; see section \ref{sec:discussion}.% The reason that \gls{DSL} was adapted but not others was due to it being more developed than the one described in the learning material, and since complex numbers was used as a dependency and not the main focus, this complexity was acceptable. \todo{Is this too much?}
\end{newtext}

\subsection{Implementation}
The clearest example of how some of the ideas presented in section \ref{sec:theory} and section \ref{sec:didactics} were used to create the learning material is the usage of \gls{GADTs}. In figure \ref{code:expression_gadts} there is an example from the learning material of how the \gls{GADTs} help emphasis the types. %For an example on how \gls{GADTs} allows for this distinction, see figure \ref{code:gadts_comparison}. 
The figure contains two (abbreviated) versions of the \cmd{Expression} datatype from the learning material. For the sake of space and ease of reading, \cmd{Expression} has been shorted into \cmd{Expr} and some lines have been removed.
In \ref{code:expression_vanilla} we see that the datatype has been declared by listing the different constructors for \cmd{Expr}. 
However, the types might not be apparent. 
\begin{modtext}
Interpreting them as different patterns on how to create \cmd{Expr}s is one way to understand them. For example, \cmd{Const} can be interpreted as for something of type x, \cmd{Const x} is of type \cmd{Expr x}. Another example is that given two expressions of type \cmd{Expr x}, adding them by putting \cmd{:+:} between them creates a new expression of type \cmd{Expr x}. 
\end{modtext}
The types should be stated explicitly, however. Implementing the datatype using \gls{GADTs} allows for explicit typing, see \ref{code:expression_gadts}.
For example, we can see that \cmd{Const} takes something of type x and returns something of type \cmd{Expr x}, just like interpreted above. 

\begin{figure}[H]
    \begin{subfigure}{0.999\textwidth}
    \centering 
    \inputminted[autogobble]{haskell}{code/expression_no_gadts.hs}
    \caption{Definition of \cmd{Expr} data type without using \gls{GADTs}. This is done by listing different constructors for the datatype with their parameters filled in, e.g. \cmd{Const a} and \cmd{Expr x :+: Expr x}, which construct expressions representing the constant function and a sum of two expressions, respectively. Note that the types of the different constructors are not explicit, possibly requiring interpretation to understand.}
    \label{code:expression_vanilla}
    \end{subfigure}  
    \begin{subfigure}{0.999\textwidth}
    \vspace{12pt} % nödlösning för att fixa spacingen
    \centering
    \inputminted[autogobble]{haskell}{code/expression_gadts.hs}
    \caption{Defintion of \cmd{Expr} data type using \gls{GADTs}. This is done by listing the different constructors (without parameters) and their respective types, e.g. \cmd{Const} and \cmd{(:+:)} are two functions of types \cmd{a -> Expr x} and \cmd{Expr x -> Expr x -> Expr x}, respectively. In this case, the types are explicit and thus likely require less interpretation.}
    \label{code:expression_gadts}
    \end{subfigure} 
    \caption{Comparison of \cmd{Expr} datatype with and without \gls{GADTs}.}
    \label{code:gadts_comparison}
\end{figure}

Another main idea that the learning material is built on is the development of \gls{DSL}s as a learning tool. For example, a \gls{DSL} to describe Linear Time-Invariant (\gls{LTI}) Systems is created, consisting of a datatype and functions between them. Explicitly building the language step-by-step is a way to help the student build an understanding of the \gls{DSL}. The DSLs were developed by identifying the function declarations. This is done by starting with a very basic case of the \gls{DSL} like \cmd{Id} and then continuing building on it and making sure it all works together. 

In order to help learning, the learning material is built using constructivistic ideas. The idea of assimilation is especially used when we connect the students previous knowledge of \gls{Haskell} through aforementioned building of \gls{DSL}s. As this is a supplementary learning material we believe it will often function in an didactically accommodating way. For example a student may learn about the shift function in the main course but not fully understand it, but when they read about it again from another perspective in our learning material they have the opportunity to relate the shift function to other prior knowledge and understand it.


%In order to conform to the project size, a selection of subject areas had to be done. Subject areas were chosen based on (in a rough order of priority) which concepts students frequently struggle with, which concepts can be effectively and accurately described using DSLs\cite{DAT326}, and which concepts can be effectively visualised and/or illustrated. 
%The reasoning behind this priority was that the most important part were to find topics that, if taught well, would help students understand what they otherwise would not, and the chosen topics should be fit to teach using DSLs. % add something about why visualisation 