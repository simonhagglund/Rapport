\section{Task}
As mentioned in section \ref{sec:purpose} (Purpose), the task will be to create supplementary learning material for Control Theory(Reglerteknik ERE103 \cite{ERE103}). 
Initially, the aim will be to make a text explaining concepts of Control Theory, see section \ref{sec:delimitation} (Delimitation) for details on which concepts. 

The text should utilise the approach taken in DSLsofMath\cite{DAT326} and implement parts of the course as a Domain-Specific Language (DSL) in Haskell---making the mathematics more explicit. Thus, a part of the task is to implement some of the tools taught in the course as a DSL, building on the work done in ''TSS with DSLs''\cite{tssarbete}.

Further, as a learning tool, the text should implement sensible didactics. This includes, but is not limited to: examples, exercises, non-jargon language, and references to any required knowledge not covered in the text.

Projects in this style have happened at least twice in the past \cite{tssarbete}, \cite{fysikarbete}. Thus, it is also likely to happen a fourth time. With this in mind, there is a secondary goal of ensuring as good documentation of the process and methods used as possible; hopefully aiding future groups as well.

A tertiary goal, subject to time restraints, is to take the information in the text and turn it into a website. See ``TSS with DSL''\cite{tssarbete} for an example of how to do this. 











\iffalse
Problem/Uppgift: (slides)

Analysen identifierar frågorna eller uppgiften som ska tas upp i projektet. 

Viktigt att göra en problemanalys eller uppgiftsanalys även om handledaren och/eller företaget redan har specificerat ett/en problem/uppgift. 

Det ”verkliga” primära problemet/uppgiften skiljer sig ofta från det som föreslagits i början av uppdragsgivaren. 

Problemanalysen syftar också till att bryta ner problemet/uppgiften i mindre och mer detaljerade delproblem/deluppgifter, vilket också leder till formulering av delsyften. 

En bra analys som identifierar delsyften vilar på användning av teorier och modeller från litteraturen.
\fi