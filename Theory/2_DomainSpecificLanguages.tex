\newcommand{\dsl}{domain-specific language}
\subsection{Domain-Specific Languages}
A \dsl{} (\gls{DSL}) is---as opposed to a general purpose language---a language that is tailored to a specific problem domain. It aims to provide an ``ultimate abstraction'' 
% \cite{techniquesforedsls,buildingdsel}
\cite[pp.~196--es]{buildingdsel} that helps improve communication \cite{fowler_parsons_2010}, and often yields better solutions to problems in their domain \cite{annotatedbibl}. Generality and domain expressiveness are largely opposing powers, where \gls{DSL}s favour expressiveness \cite{mernik_heering_sloane_2005}.
\gls{DSL}s are often considered a subset of programming languages, but this notion is not accepted by everyone \cite{annotatedbibl,fowler_parsons_2010,dsvl}. Arguably, any language that describes a domain with more specificity than a general purpose language might be considered a \dsl{} \cite{dsvl}. Commonly known \gls{DSL}s include \LaTeX, SQL, and HTML \cite{annotatedbibl}. 

In addition to these programmatic \gls{DSL}s, one could argue there are other kinds of \gls{DSL}s using other mediums, such as \gls{DSVL2}s  (\gls{DSVL}).
\gls{DSVL}s are often different types of diagrams and may also have an accompanying programmatic \gls{DSL} \cite{dsvl}. 

In this report the term \gls{DSL} refers to the more generally accepted notion of a \gls{DSL} as being a programming language or other executable specification language, except for when explicitly referring to \gls{DSVL}s or other kinds of \gls{DSL}s.


%``DSLs trade generality for expressiveness in a limited domain.'' \cite[p.~317]{mernik_heering_sloane_2005}.


\subsubsection{Embedded Domain-Specific Languages}\label{sec:edsl}

Implementing \gls{DSL}s from scratch takes a lot of time and effort, and it is therefore common to implement them within another (general-purpose) language \cite{buildingdsel}, called a host language \cite{techniquesforedsls}. This kind of \gls{DSL} is referred to as an embedded DSL (\gls{EDSL}). These are relatively easy to both implement and extend as they use already established languages. 
A secondary advantage of this is that the host language can  provide general-purpose capabilities to the \gls{DSL}, if need be \cite{annotatedbibl}.
Functional languages such as \gls{Haskell} are popular as host languages for their capabilities of defining custom operators, using higher-order types, as well as overloading \cite{techniquesforedsls}.

\begin{figure}[ht]
\begin{subfigure}{.549\linewidth}
        \centering
        \inputminted[autogobble]{haskell}{code/data.hs}
        \caption{Deep \gls{DSL} with \gls{GADTs} language extension.\newline}
        \label{code:gadts_dsl}
\end{subfigure}
~
\begin{subfigure}{.449\linewidth}
        \centering
        \inputminted[autogobble]{haskell}{code/data_no_GATDs.hs}
        \caption{Deep \gls{DSL} without language extensions.} 
        \label{code:vanilla_dsl}                    
\end{subfigure}
\begin{subfigure}{.549\linewidth}
    \centering
    \inputminted{haskell}{code/class.hs}
    \caption{Declaration of a \gls{Haskell} class.\newline} 
    \label{code:class}
\end{subfigure}
~
\begin{subfigure}{.449\linewidth}
    \centering
    \inputminted[autogobble]{haskell}{code/shallow_dsl.hs}
    \caption{Shallow embedding of the \gls{DSL} in \subref{code:gadts_dsl} and \subref{code:vanilla_dsl}.} 
    \label{code:shallow}
\end{subfigure}
\caption{Three different implementations of a small \gls{DSL} in \gls{Haskell} are shown in figures \subref{code:gadts_dsl}, \subref{code:vanilla_dsl}, and  \subref{code:shallow}. Two of these, \subref{code:gadts_dsl} and \subref{code:vanilla_dsl}, are deep \gls{DSL}s, while the third, \subref{code:shallow}, is a shallow \gls{DSL}. Figure \subref{code:class} shows the likeness in declaration of classes and \gls{GADTs} data types.}
\label{fig:GADTs}
\end{figure}

In figure \ref{code:gadts_dsl}, an example of a small \gls{EDSL} is presented. When the syntax of the \gls{DSL} is encoded into the algebraic data type as constructors, the \gls{DSL} is called a deeply embedded \gls{DSL}, as opposed to a shallow embedding as is shown in figure \ref{code:shallow}. Shallow embeddings do not have the same clear distinction between syntax and semantics. The usage of this kind of \gls{DSL} will always result in instant semantic evaluation, since the syntax and semantics are embedded in the same structure. In contrast, a deep embedding---where the constructors encode the structure (syntax) and the semantic evaluation separately---will determine the result of that structure at some later point in time, and may instead manipulate the syntactic structure by itself. %Different semantics may be implemented for the same syntax.
%% Remove or expand this line.
%%% This was moved %%%
%\todo{Should this be moved to discussion? There are no sources for this and it is very "speculative." /T}The native differentiation of syntax and semantics in deeply embedded DSLs maps quite well onto differentiating technical frameworks' notation and the evaluation of that structure, and many domain-specific terms could be directly represented as constructors in a Haskell data type. For instance, definite integrals, line integrals, and surface integrals could be represented as three or more constructors in a data type for generalised integrals. These could then be evaluated in a number of ways, including syntactic simplification and direct semantic numeric approximation.


There are many language features of \gls{Haskell} that simplify the implementation of embedded \gls{DSL}s. %, which in conjunction with language extensions---i.e. extensions that teach/augment the compiler how to type infer or adds language constructs [Ref]---offers a strong framework for dealing with DSLs. 
The most noteworthy extension for this project is \gls{Generalised Algebraic Data Types} (\gls{GADTs}) which adds \gls{Haskell} syntax that facilitates declaring algebraic data types much like one would declare a \gls{Haskell} instance or class.

In the code in figure \ref{code:gadts_dsl} a small \gls{DSL} over addition is implemented, where \gls{GADTs} allows class-like syntax like in \ref{code:class} when declaring a new algebraic data type.
Figure \ref{code:vanilla_dsl} shows a version of figure \ref{code:gadts_dsl} without the \gls{GADTs} extension. These three snippets precisely describe the same behaviour in \gls{Haskell}, but \gls{GADTs} offers a type declaration syntax closer to that of type theory---making these types more easily readable. 


%...
%``DSL development is hard, requiring both domain and language development expertise. Few people have both.'' \cite[p.320]{mernik_heering_sloane_2005}
%...