\section{Discussion}\label{sec:discussion}



The use of \gls{DSL}s and \gls{Haskell} to teach other subjects is---as previously discussed in section \ref{intro}---an interesting approach that has proven itself useful in various mathematical domains for a number of reasons. %Here we will evaluate our project and discuss what we have learned. 
In this section, the project will be evaluated and discussed in detail.

%\todo[inline,color=other]{This section is by nature heavily WiP: full discussion cannot be written until conclusion of project.}


\subsection{Evaluation of the Learning Material}
The idea of using types and \gls{EDSL}s to teach mathematical subjects such as \gls{controltheory} relies on the idea of instrumental scaffolding, discussed in section \ref{sec:scaff}. Students that are comfortable with programming---and especially functional programming---can use this knowledge as cognitive scaffolding to have an easier, more intuitive way into new subjects. To enable this, students must have a learning material or guided instruction that uses this scaffolding in order to teach the subject matter, which is what the learning material developed for this report does. The material also has the advantage of being able to instruct a narrower group of students---a teacher has to accommodate everyone's needs simultaneously, whilst the material allows itself to have a much smaller target audience. 
The intent is that this will be enough for a student to enter a \gls{ZPD}, and after having used the material have the cognitive scaffolding required to be able to continue their learning of \gls{controltheory} in a non-computer science environment.


For the scope of this project, it is not possible to cover all of the course content of the targeted course. Some essential parts had to be be prioritised over others, some areas are more central for the understanding of \gls{controltheory} and fundamental for further learning than others, and some areas are not as well suited for programmatic \gls{DSL}s as others. 


\begin{newtext}
Section \ref{sec:process} mentions which domains described in the learning material have already had \gls{DSL}s developed in previous projects \cite{dslsofmath,tssarbete}, and whether or not the \gls{DSL} is used in the learning material, however, the reasoning behind whether to reuse a previously developed \gls{DSL} or to make a new one is not explained. 
When investigating the \gls{DSL}s from the previous projects, all of them were developed in ways not congruent with the envisioned description in the learning material. For example, the section on Laplace transforms heavily leans on the tabular approach, which in our opinion was served by allowing the creation of expressions that could be transformed. The approach taken by \gls{DSLsofMath} is to not implement the Laplace transform, but to show how it can be used to calculate the solution to a differential equation. This approach is more similar to the ordinary mathematical approach to using the Laplace transform, and thus our implementation could fill what we regarded as an empty niche. 
Similar thoughts guided the decisions on whether to keep the 
\gls{DSL}s developed in previous projects,
%other \gls{DSL}s previously developed, 
with one exception: in later parts of the learning material, the \gls{DSL}s developed relied on an implementation of complex numbers. In order to maximise utility for these \gls{DSL}s, a more extensive \gls{DSL} for complex number would be better. Thus the \gls{DSL} for complex numbers developed by ``TSS with DSLs'' \cite{tssarbete} was adapted to fit our \gls{DSL} and used as a dependency in later \gls{DSL}s. 
%overlap with previous projects and describes whether a new \gls{DSL} was created 
\end{newtext}

\subsubsection{Prerequisites}

A significant part of the learning material is taken up by content that is related to the prerequisite knowledge of the course, and one might argue that this puts the focus of the material on what is not of the subject at hand.
But, as mentioned in section \ref{backgroundERE}, many students do not have enough prerequisite knowledge to take full advantage of the course. For this reason, an introductory part was included to in part serve as aid for some of this missing or unpractised knowledge. As to exactly what prerequisites this section would best address, there are many possible answers. The course plan \cite{ERE103} for the course lists complex numbers, linear algebra, Taylor series, ordinary differential equations, the Laplace transform, and knowledge of basic relations from physics, as prerequisites---any of which would be good candidates for this section. Additional concepts could also be argued are lacking in the students' repertoire of knowledge, such as line integrals which are only briefly discussed in the course before applying them to Nyquist diagrams. 

Of all of these areas complex numbers, integrals, and Laplace transforms were chosen to be included in the material. Complex numbers are arguably, for many students, probably of the least concern, since we can assume they have this knowledge from their secondary education. However, the section on complex numbers in the learning material has an additional purpose of serving as a refresher on \gls{Haskell} syntax and familiarising the reader with the \gls{DSL} approach of thinking about mathematical domains, and would therefore benefit from describing an area already familiar to the reader. 

Integrals are presumably also rather familiar to the intended reader, whose section in the learning material has a similar purpose to that of the section on complex numbers. What is different from that section, however, is how the respective subjects are traditionally taught in relation to their types. The type of the integration operation is a higher-order type, which is generally not taught when learning about integrals---likely giving a less rigorous understanding of the operation. As the understanding of types is central to this \gls{DSL} approach of learning, integrals were deemed important that they be included in the introductory part of the learning material. Furthermore, integrals may be seen as very essential to subsequent parts in the learning material. 

Two different ways of calculating integrals were presented in the learning material. A potential downside of this structure is that once students have understood the first solution, they may lose interest in the latter. For our purposes, leaving it up to the students how deep they want their understanding to be is acceptable, as understanding only the numerical solution should be sufficient for later sections. Another thing to consider is the choice of using a numerical way to calculate integrals. In the course you mostly solve integrals analytically. It can also be argued that numerical solutions do not give very good understanding and intuition for the technique. On the other hand, not all integrals can be solved analytically and it is also important that the student understands how the integrals are computed in that case. We do however provide two ways of calculating integrals which means we get the benefits of both approaches.\todo{Didactics? Different people constructs knowledge in different ways such that multiple representions of the same abstract mathematical concept is offers diffrent people diffrent oppertunities...}

Lastly among the prerequisites, the Laplace transform was explored in the learning material. This section comes somewhere between introductory content and main content of the subject, as the Laplace transform is quite heavily and directly used in many of the areas addressed in the course. Tables of different known Laplace transforms and combinations of them are used when learning how to compute the Laplace transform. The same technique is used in the implementation described in the learning material.
The amount of code required might be intimidating to the reader at first, but a pattern quickly becomes clear. 
This is probably beneficial to the students because this shows that the Laplace transform always uses the same methods.



\subsubsection{Core Course Subjects}
\iffalse
\todo[inline,color=other]{Write about the more central parts included, and if we should've included more.\\
*Tillståndsmodeller,\\
*Pid-regulatorer,\\
*Problematik med nyqvist,\\
*Bode diagram,\\
*Stabilitet och andra metoder för att kontrollera detta,\\
*Linjärisering,\\
*De vi skrivt om: LTI, TF
}
\fi


Transfer functions are used in most of control theory and the course has a clear focus on linear time-invariant systems, so these parts were deemed important and take up a significant part of the material. 

The Laplace transform, while not easy, was well suited to implement as a DSL. 
Some of the more involved areas of \gls{controltheory} proved quite difficult to construct \gls{DSL}s for. For example, the Nyquist criterion, which uses numerical approximations and intuitions that are hard to represent with \gls{DSL}s.
%For more complex areas of \gls{controltheory} it was quite tricky to construct \gls{DSL}s. 
%For example the Nyquist criterion was an area we struggled a lot with. 
%It uses a lot of numerical approximations and intuition which is hard to represent with \gls{DSL}s. 
The implementation has begun, but is not complete.
%We have finally started on a way to implement it but it is not complete. 
Bode diagrams was initially intended to be included in the learning material, but similar problems hindered its development. 
%The Bode diagrams which we first aimed to cover in the learning material were hindered by similar problems. 
We believe that it is possible to construct a good and useful DSL for Bode diagrams, but it was not possible in the time frame of this project.

The other core course subjects such as PID-regulators were not implemented due to their dependency on other areas not implemented, such as Bode diagrams. 
%The other core course subjects such as PID-regulators were not covered mainly due to the fact that the previous areas such as the Bode diagrams not being complete. 
This is however something that can be worked more upon in the future.


\iffalse
\begin{itemize}
% \item This learning material have not been evaluated in any way by anyone outside the project group. This means we have no objective way to evaluate the material's quality. What we can discuss is what we feel are the largest areas that can be improved upon. : passive learning material, we need the project to be more finished before we use this part

    \item \sout{The choice of \gls{Haskell}: jämföra både hur \gls{Haskell} stödjer den matematiska aspekten och hur \gls{Haskell} artar sig som värd för inbäddade språk - There are a lot of \gls{Haskell} specific syntax that can make it difficult if you have not used \gls{Haskell} recently. Perhaps you could even use pseudo code, you lose the ability to actually run the program but it would make for a much more readable text that does not require any knowledge of a specific programming language. }



\item \sout{The choice of English as the language of the course material}: 


\item \sout{Does not cover the entire course (choice of subjects?)}

\item The way the material is presented (\gls{didactic}s)

\item Discuss for some particular areas why we did not include them in the material. Bode (not mentioned before this point (I think)); \sout{contour integral}
\item \sout{Website?} 
\item \todo{överstökat nu? (Se första stycket i \ref{sec:useofhaskell})}Motivate use of \gls{Haskell}: students of the course have used \gls{Haskell} before; vi bygger vidare på \gls{DSLsofMath}
\item Possibly discuss us not using different semantics (because we don't, right?) for the same deep embedding in order show how the same structure can be viewed in different ways
\end{itemize}
\fi

%A secondary goal, subject to time restraints, was to turn the learning material into a website. As websites can be more interactive than just text, turning the learning material into a website can enable the exercises to take answers as input. This can activate the student and help further learning. 



\subsubsection{Exercises}
There are a few select exercises included in the learning material. It would have been useful to have more exercises included in the material, but this ended up in second priority to other aspects of the material such as \gls{DSL}, text design, and subject research. Optimally, more exercises would have been included in order to allow students to interact with the learning material to a greater degree, yet \gls{ERE103} has an abundance of exercises. Exercises differs somewhat from text describing correlations and how the concepts in a field fit together. The main difference being that exercises teaches students what problems are typical of the field, and how to apply the theory to those kinds of problems. 


\subsection{The Use of Haskell as a Host Language}\label{sec:useofhaskell}

Of the various ways of implementing a \gls{DSL}, embedding it in a host language, as discussed in section \ref{sec:edsl}, requires some of the least effort and is easy to extend and modify. Functional languages are popular as host languages as they are often very expressive and flexible \cite{techniquesforedsls}. Operators are often overloadable which opens up the possibilities for making even more tailored \gls{DSL}s---making implementation of \gls{DSL}s for many mathematical domains especially natural. \gls{Haskell} is one such language, that, as mentioned in \ref{sec:edsl}, has become popular for making embedded \gls{DSL}s \cite{techniquesforedsls}, which is in part why it was deemed suitable for the project. In the case of \gls{Haskell}, implementing types as instances of type classes takes the role of overloading operators. 
Previous work in the same area \cite{tssarbete,fysikarbete,Ionescu_2016} have also used \gls{Haskell} as a host language, and the students of the \begin{modtext}CSE \end{modtext} program---in which the targeted \gls{controltheory} course resides---at Chalmers has a course about \gls{Haskell} in their first year, so this choice came quite naturally.

One possible drawback to embedding \gls{DSL}s in \gls{Haskell} for mathematical domains is the strictly one-dimensional nature of such code. Mathematical notation often utilises vertical alignment---writing above and below other mathematical symbols---to convey information and make the syntax easily readable. For example, the notebook-style application Wolfram Mathematica \cite{Mathematica} allows for this kind of two-dimensional syntax input via the Wolfram Language. This kind of input, although very useful for its purposes, was not deemed as suitable for this project as the syntax is much closer to the original mathematical syntax, making it less familiar to those comfortable with regular program code. In order to utilise this familiarity we chose to use \gls{Haskell} to tailor the \gls{DSL}s to computer science students. 

Using \gls{Haskell}---or any other programming language---as a host language for \gls{DSL}s has a risk of alienating a significant portion of the target audience to some extent, as they may not be very familiar with the language. It could thus arguably be a good idea to be restrictive with language-specific implementation, or even use pseudo-code. As has been done in previous work \cite{dslsofmath}, \gls{Haskell}'s type checker could be relied on without specific implementations of functions, and by these means still serve as effective learning material. Arguably, there could still be beneficial for those familiar with the language used to study the implementation to some extent, and those unfamiliar need not direct their attention to implementation details. The learning material developed relies on this latter view, which is beneficial to disclose to the reader so that they have appropriate expectations and use the material in an effective way.

As a secondary kind of \gls{DSL}---a \gls{DSVL} of block diagrams native to control theory---has been used. These are meant to serve as a comfortable middle-ground between \gls{controltheory} and the \gls{Haskell} \gls{DSL}s presented in the learning material. Discussing the types of the components involved in the diagrams can provide a more rigorous understanding of the diagrams' relation to \gls{LTI} systems, transfer functions, and impulse response, among other things. Relating these diagrams to the \gls{Haskell} \gls{DSL}s presented makes for a clearer connection between the different representations. 



\subsection{Medium of Presentation}%Language of Instruction}
The material was chosen to be presented in a textual form, and specifically as a PDF. This is in line with a lot of other course material, and as the content is mostly code based, and previous work has been presented in the same way, it was a natural choice to present it in text form. A website with interactive exercises was long considered, but did not fit the time constraints of the project and was deemed less important than other aspects such as the content scope of the material. Other mediums could have been suitable as well, such as interactive tools with a \gls{DSVL} focus, or video lectures. The reason these were not chosen for this project is that these would have taken a lot more time and effort away from expanding the content scope and refining the presentation. A textual presentation was deemed the most time-efficient, assessed to yield the most amount of valuable content, and have the least amount of hurdles in the way of reaching our intended goal.

The choice of language to present the material in was not entirely trivial, as the course that the material specifically targets is held exclusively in Swedish \cite{ERE103}. The students in question are assumed to have an English language proficiency adequate to effectively use the material. However, since a lot of terminology differs between the languages, the material risks becoming less accessible to the attendants of that course, and terminology might get confused. On the other hand, presenting the topic in English opens up for a larger audience that may access the learning material; both for learners of \gls{controltheory} and for those who wish to conduct further research on the matter. 
In addition, this may work as a middle ground, possibly helping students learn the English terminology, allowing them to read a broader literature on the subject.
 


\subsection{Future Projects}
There are three main ideas for projects that can build on our work. The first would be a project that evaluates our learning material. This could be done by testing it on a group of students and evaluating whether it improves their understanding of the subject \begin{modtext}or \end{modtext} improves their performance on the exam. We were unable to test it on students taking the targeted course as the course runs in fall and this project took place during spring.

The other project would simply be a continuation of this project's material. 
As mentioned before, the learning material is not complete in relation to the scope of the targeted course, or the scope of control theory as a whole, and there are several areas of the learning material that could be improved. As such, a similar project could be performed where the groundwork has already been laid, and could therefore be much further developed than this project has. Such a project could explore areas of \gls{controltheory} that was not covered by this project, or add more material to the existing parts such as exercises for the reader to do. 

Finally, the third idea for a project that can build on our work is a project that develops a learning material for a different course or subject. The same methodology using \gls{DSL}s can be applied to many different areas and can be beneficial to learning.

\subsection{Ethics and Utility}
The most important societal aspect to mention of this project is the fact that this is a free supplementary learning material which can benefit students and help them in their learning. The impact of this, of course, depends on how many students use the material and the quality of this material. A possible negative economic effect of this is that students may choose to use this instead of the recommended course literature. However we do not think our project will have a very large impact on the sales of \gls{controltheory} literature as the material is not complete and does not cover all the areas covered in the \gls{controltheory} course literature. 

There are some other ethical considerations concerning this project. One of which concerns the fact that this project does not empirically establish the effectiveness of the learning material, but is instead entirely reliant on parts from the theory of \gls{didactic}s explored in this report. Presenting the material as an effective alternative approach may not be ethical in case it is not as useful to this group of students as presumed, or even make the subject more confusing. Therefore, the learning material should arguably be clearly presented as untested and/or experimental.

%The goal is to develop learning material and therefore it is important that it actually is helping the students understand the topic better. Not, for example, just make it easier to pass the exam without understanding the material. 
Furthermore, as the material developed is meant to help students understand the subject better, although not that likely, the risk for the material to, for instance, make it easier to pass the exam without understanding the material, should be considered.
%This will probably not be a problem because the type of material intended is only meant to make things more intuitive, but it is important to keep in mind.
This could be an issue if the learning material focuses too much on the programmatic side without emphasising the concepts and physical aspects of \gls{controltheory}. It may also be important that it be emphasised that the learning material cannot replace any part of the targeted---or any other---course, and only exists as supplementary material.

It is also important that the material is available to everyone---not only a select few. This is assured by making all of the work available on GitHub and also written in English. The aim is also to have the course material published on the \gls{ERE103} website, which means it would be easily available for all students in the course.

%Control systems are used everywhere, even in military industry whether it be defensive or offensive technologies. The nature of these industries ensures that people working in them need to take a stand on the ethical aspects of these professions. Concerning this topic, a section discussing the ethics of control systems and possible dispositions knowledge in these areas could put us in will be included in the project.

%%A clear ethical aspect of learning about control systems i  
%%Another ethical dilemma is the use of control systems is the use of them 

%wrong
\iffalse
Besides the utility of the project, it is not likely that this project would harm or impact society in any negative way as such. As stated earlier the content is very specifically to demystify mathematics and concepts therein for students of control theory. 
\fi

%On a different note there are tools that are practical and helpful while construction constructs of \gls{controltheory}, of which of Matlab - Control System Toolbox \cite{matlab_control} seems most wildly used. However Matlab is high level and offers close to no help with mathematical intuition required to be able to understand and solve common \gls{controltheory} problems, however this is the focus of this project which increases the utility of this project. 


%Det finns två dimension här, dels att programmering är en enkel väg in i reglerteknik, men också kanske lite oväntat det omvända, att regler tekniken är en väg in i programmeringen. 

%Läser jenkis som hävdar att en stor andel av de som går program som innehåller programmering eller där det utgör en stor del av bachelor eller master, egentligen vill undvika eller inte är motiverade att lära sig koda. \cite{jenkins_2001}

\iffalse

\subsection{Instructional Scaffolding & Zone of Proximal Development}
\textbf{Point of this section}: Present ZPD
\textbf{end goal}: Claim justification for what the product aims to do.
\newline

Is the learning zone where a student is capable of learning with the help of others.. (the product aimes to push student into this zone by relating stuff the student knows to what the student does not yet fully grasps \cite{vygotski_1978}.

During any course students need support structures meant to provide students with aid as to allow them to reach new knowledge, 
When undertaking any new course


Scaffolding is the support structure from teachers for the student.. 

This is what the product aims to be a part of.
``ZPD and scaffolding complement the need
for activation of prior knowledge to understand programming concepts. It is relevant in
rich learning environments which afford learners the ability to build a community with
their peers and teachers."  \cite{wood_bruner_ross_1976}



\textbf{Point of this section}: Provide didactic context. What it is. (ZPD and Scaffolding covers it)
\textbf{end goal}: Account for full teaching environment. 
\newline

\subsection{Motivation}
\textbf{This section can be removed maybe?}

\textbf{Point of this section}: Present theory about motivation.

\textbf{end goal}: Claim that motivation is most important but abstract.
\newline

When a student learns programming, the student learns a practical skill, but it is no easy feat. It requires students to engage with coding whether there is credits to be made or not, \cite{jenkins_2001} to practise iteratively. The student must accept and get used to a style of teaching where the theoretical and practical intertwines in tasks that must be considered a good idea \cite{jenkins_2001}. 
Most things in programming lacks a real life example of application of programming concept \cite{dunican_2002}, which forces CS-students to get used to a methodology of learning, which no small part consist of writing code to internalise by exploration.

In control theory we really speak of the parameters of the mathematical functions we applies, when designing \gls{DSL} in \gls{Haskell}, this instead is a requirement. 

\fi

\iffalse 
\subsection{ideas about content}

\begin{itemize}
    \item Here we should write about what could have been done better.
    
    -planning, strict structure vs agility, early study section vs continuous learning \& iterative design
    
    -start small, not try to tackle everything immediately 
    


\item Ethics goes here and it is important that it is a natural part of the text. We can use a lot of inspiration from the planeringsrapport text but i think it will be good to start fresh to get an improved structure and make it more natural.

-Availability:why we wanted the material to be in english, the importance of it being available on github. 

- consideration of how it can impact students learning, will they learn it the ``wrong'' way(use haskell on the exam? will it actually contribute to a deeper understanding?. Here we will write what our intentions with the material was and why we think it is unlikely that it will be used the wrong way

-we said we would write a part about the ethics of control systems in the material. If we do that mention that we did it and why we did it.
\end{itemize}
\fi 


% skriva om haskell,  vilka alternativ som finns och om det finns nåt som kanske hade lämpat sig bättre

% koppla mycket till didaktikdelen, läsa igenom och kolla att allt hänger ihop. t.ex skriva att vi applicerar den kunskapen om didaktik när vi använder haskell för att förklara regler.

% write something about why we have the complex numbers part and connect it to the theory where we say that lacking prerequisites are a problem

%english vs swedish

%prata lite om förkunskaper