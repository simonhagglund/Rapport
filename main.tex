\documentclass%[final]%removes todos
{article}
\usepackage[left=4.5cm,top=4.5cm,right=4.5cm,nofoot]{geometry}% Distance to edges, change so it looks good and we get a good number of pages
\setlength{\footskip}{40pt}
\usepackage[dvipsnames]{xcolor}     % some nice colors
\usepackage[swedish, english]{babel}% Tells latex that we write in english, so it follows english rules. Swedish for sammandrag.
\usepackage[utf8]{inputenc}         % MISSING
\usepackage{graphicx}               % more options for includegraphics, not sure it's currently necessary 
\usepackage{float}                  % MISSING
\usepackage{comment}                % MISSING
\usepackage[style=ieee
  , backend=biber]{biblatex}        % reference manager 
\usepackage{array, booktabs}        % prettier tables, I think?
\usepackage[TS1,T1]{fontenc}        % MISSING
\usepackage[obeyFinal]{todonotes}   % inline/margin todo-labels
\usepackage{ifthen}                 % can't find what this does? 
\usepackage{tocloft}                % related to table of contents somehow?
\usepackage{hyperref}               % hyperlinks in text
\usepackage{minted}                 % Highlights code.
\usepackage{bashful}                % dynamic word count

\usepackage{subfig}                 % Adds subfigures in figures.
\usepackage{caption}
\usepackage{csquotes}               % supposed to ensure quotes are right depending on context - csquotes = contex sensitive quotes 
\usepackage[normalem]{ulem}         % strikethrough; normalem fixes the underlined references
\usepackage{subfiles}               % Ladda denna sist.
\usepackage{amsmath}
\usepackage{pdfpages} %för att bifoga produkten

\usepackage[toc]{glossaries}%för ordlista
\makeglossaries
\loadglsentries{ordlista.tex}


\bash
texcount -sum -1 abstract.tex
\END

\hypersetup{
    colorlinks=true,
    linkcolor=blue,
    filecolor=magenta,      
    urlcolor=cyan,
    citecolor=red,% open for suggestions, I just hurt my eyes on the bright green
}

\DeclareCaptionFont{blue}{\color{LightSteelBlue3}}
\addbibresource{refs.bib}
\setlength{\marginparwidth}{4.3cm}
\newcommand{\foo}{\color{LightSteelBlue3}\makebox[0pt]{\textbullet}\hskip-0.5pt\vrule width 1pt\hspace{\labelsep}}

% ToC depth
\setcounter{tocdepth}{4}
\setcounter{secnumdepth}{4}
\newcommand{\subsubsubsection}[1]{\paragraph{#1}\mbox{}\vspace{2mm}\\}


% Adds choice for hiding stuff when we are in a writing phase. 
\newboolean{DELOPMENT}
\setboolean{DELOPMENT}{false}   

% quick color names for ease of remembering
\colorlet{urgent}{red}
\colorlet{fixme}{orange}
\colorlet{lessurgent}{blue!20} 
\colorlet{other}{yellow!30} 


% some commands for semantic tagging
\newcommand{\cmd}[1]{\mintinline{#1}} % to typeset commands 
\newcommand{\course}[3]{``#1'' (``#2'' #3 \cite{#3})} % arguments, in order: english name, swedish name, course code


% Educational Language of Control Theory?
% Teaching Control?
% Domain Specific Education, Control Edition?
% Learning the Language of Control?
% Learning Material for Control Theory with Domain-Specific Languages?
% Demystifying syntax and semantics of control theory.  
% Controlling Education through Domain Specific Languages 
\title{Demystifying the Syntax and Semantics of Control Theory
 \vspace*{5.5mm}
 % (Using Domain-specific Languages?)
 \\ {\Large Developing course material using \\ Domain-Specific Languages} 
 \vspace*{85mm}
 }
  
\author{

    Filip Nylander \and
    \vspace*{0.001pt}
    Jakob Fihlman \and 
    Christian Josefson \and 
    Elin Ohlman \and 
    Tommy Räjert \and 
    Simon Hägglund
    }
\bigskip
\date{\vspace*{5mm} Spring - 2020} % unless anyone comes up with anything better 

\begin{document}

\maketitle
\thispagestyle{empty}


\setlength{\marginparwidth}{2.6cm} % fixes todonotes margin problem
\ifDELOPMENT
Proposed colour coding for todo notes. TR
\todo[color=gray!10]{Colour coding:}
\setlength{\marginparwidth}{4.3cm}
\todo[color=urgent]{Urgent comments}
\todo[color=fixme]{Fix-me comments}
\todo[color=lessurgent]{Less urgent comments}
\todo[color=other]{Other highlights}
\fi

\newpage

\begin{abstract}
    \subfile{abstract} 
\end{abstract}
\begin{otherlanguage}{swedish} 
    \begin{abstract}
     \subfile{sammandrag} 
    \end{abstract}
\end{otherlanguage} 
    
\iffalse % to save it
\centerline{\bf Abstract}
    ~ \\ \noindent
    \subfile{abstract}~\\~\\~\\
\centerline{\bf Sammandrag}
    ~ \\ \noindent
    \subfile{sammandrag}~\\ ~
\fi 



\newpage
\renewcommand*\contentsname{Table of Contents}
\renewcommand{\cftsecleader}{\cftdotfill{\cftdotsep}}

\tableofcontents
\newpage
\setcounter{page}{1}
\pagenumbering{Roman}
\setcounter{page}{1}
\pagenumbering{arabic} 

\printglossary
\newpage
\subfile{introduction}
\newpage
\section{Theory}
This section is dedicated to the theory behind our report. 

\subsection{Control Theory Course ERE103}\label{backgroundERE}
The Chalmers course \course{Reglerteknik}{Control Theory}{ERE103} had a failure rate of approximately 50~\% in 2019 \cite{ere103_minutes}. 
Not only is this an unreasonably high number in its own right,
\begin{modtext}
the course is also mandatory for students at the \begin{modtext}
CSE 
\end{modtext} \footnote{``Computer Science and Engineering'', ``Datateknik'' in Swedish} programme. Thus, having a high rate of failure makes the \gls{controltheory} course a roadblock for many students. 
\end{modtext}
%Not only is this an unreasonably high number in its own right, it is also a mandatory course for the ``Computer Science and Engineering'' programme. 
%The high failure rates makes the \gls{controltheory} course a roadblock for many students \cite{Jansson_2019}.

\iffalse % OLD, was repetitive, new text below
Sometimes students struggle with other parts than the course content itself. For example there are parts of \gls{controltheory} that utilise the Laplace transform to solve differential equations, yet it is possible that students spend more time trying to understand what the Laplace transform does and less on understanding what the differential equations mean and how to use the results. In an interview \cite{tssarbete}, the examiner of \gls{ERE103} speculated that the main problems the students have are based on not having enough experience in mathematical thinking. For example having trouble understanding the connection between abstract mathematics and the real-world thing being modelled. This may cause students to spend a majority of their time and energy on understanding the mathematics and not enough on understanding the material taught in the course.
\fi

\begin{newtext}
 In an interview \cite{tssarbete}, the examiner of \gls{ERE103} speculated that the main problem that students have is based on not having enough experience in mathematical thinking. As an example, they may have trouble understanding the connection between abstract mathematics and the real-world thing being modelled. This may cause students to spend a majority of their time and energy on understanding the mathematics and not enough on understanding the material taught in the course. For a more concrete example there are parts of \gls{controltheory} that utilise the Laplace transform to solve differential equations, yet it is possible that students spend more time trying to understand what the Laplace transform does and less on understanding what the differential equations mean and how to use the results.
 \end{newtext}

\begin{modtext}
According to the course evaluation minutes of the 2019 instance of \gls{ERE103} \cite{ere103_minutes}, they have not been taught the prerequisites of the course to a satisfactory level. This is in spite of the course beginning with a short review of said prerequisites.
\end{modtext}
\iffalse
\begin{modtext}
The minutes of the course evaluation of the 2019 instance of the course \cite{ere103_minutes} mention
\end{modtext}
%The minutes of the most recent course evaluation \cite{ere103_minutes} 
that although the course starts with a review of the prerequisite material needed, students have reported that they have not been taught some of the prerequisites to a satisfactory level;\fi
For example, the previously mentioned problems with using the Laplace transform to solve differential equations. 
The syllabus \cite{ERE103} of the course explains the necessary prerequisites thusly:
``Basic concepts of mathematics that must be mastered before the start of the course are: Complex numbers, Linear algebra, Taylor series, Ordinary differential equations, Laplace transform.'' 

%% Amongst computer science students, faculty staff and alumnus the course is commonly refereed to as the hardest course of the BCD in CS, a fact that reinforces that view is that is the fail rate. 


%Denna text kan läggas på en annan del om det behövs.
%% The creators of \gls{DSLsofMath} have found that students attending \gls{DSLsofMath} had a greater chance of passing courses requiring a lot of prior knowledge in mathematics the following year \cite{Jansson_2019}. The same approach could be applied to \gls{controltheory} by creating one or more \gls{DSL}s. As such, students could benefit by learning about \gls{controltheory} from a programmer’s point of view seeing as this has worked for a different subject and that \gls{controltheory} is an area many students struggle with.

%There have been previous projects where the approach of using DSLs has been used: one of these projects, which resulted in a BSc thesis \cite{tssarbete}, explored DSLs' uses in the course ``Transforms, Signals and Systems'' (Transformer, signaler och system SSY080 \cite{SSY080}). In this report, this project will be referred to as ``TSS with DSLs.''
%Another BSc thesis \cite{fysikarbete} examined the course ``Physics for Engineers'' (Fysik för ingenjörer TIF085 \cite{TIF085}). This will be referred to as ``Physics with DSLs.''

%The same methods used in the aforementioned projects and course should be useful for providing supplemental learning material for the Control Theory course as well. Especially so for students with a background in functional programming. The work carried out in ``TSS with DSLs'' will be extra useful since the courses SSY080 and ERE103 share many common traits.

%%%%%%%%%%%%%%%%%%%%%%%%%%%%%%%%%%%%%%%%%%%%%%%%%%%%%%%%%%%%%%%%%%%%%%%%%%%%%%%%%%%%%%%%%
%                                                                                       %
% Page End                                                                              %
%                                                                                       %
%%%%%%%%%%%%%%%%%%%%%%%%%%%%%%%%%%%%%%%%%%%%%%%%%%%%%%%%%%%%%%%%%%%%%%%%%%%%%%%%%%%%%%%%%

\iffalse
\section{Terminology}
\subsection{Domain-specific languages} \label{sec:DSLs}
A DSL is a specialised language for a particular domain, in contrast to general purpose languages which are generalised across domains. There are a multitude of DSLs in the world, of which HTML, \LaTeX, and Matlab \cite{mernik_heering_sloane_2005} might be most renowned.

% In Haskell, DSLs are implemented in the meta language as deep, shallow, or intermediate 
% embedding, and are a very powerful way of both syntactically codifying a domain, as 
% well as semantically providing the meaning of syntax and desired behaviour.
% partly describing a problem, the syntax, and also solving it, the semantics.

Anteckningar(Slids):

Vad är ämnet/problemet som ska undersökas? 
Varför har ämnet/problemet uppkommit? 
Varför är det ett relevant eller intressant ämne/problem? 
För vem? 
Kan det specifika ämnet/problemet relateras till en mer generell diskussion?
\fi
\newcommand{\dsl}{domain-specific language}
\subsection{Domain-Specific Languages}
A \dsl{} (DSL) is---as opposed to a general purpose language---a language that is tailored to a specific problem domain, aiming for an ``ultimate abstraction'' \cite{techniquesforedsls,buildingdsel}, that helps improve communication \cite{fowler_parsons_2010} and often yields better solutions \cite{annotatedbibl}. Generality and domain expressiveness are largely opposing powers, where DSLs favours expressiveness \cite{mernik_heering_sloane_2005}.
DSLs are often considered a subset of programming languages, but this notion is not generally accepted \cite{annotatedbibl,fowler_parsons_2010,dsvl}. Arguably, any language that describes a domain with more specificity than a general purpose language would, should be considered a \dsl{} \cite{dsvl}. Commonly known programmatic DSLs include \LaTeX, SQL, and HTML \cite{annotatedbibl}. In addition to these programmatic DSLs, according to some, there are other kinds of DSLs using other mediums, such as visual domain-specific visual languages (DSVL) \cite{dsvl}. %, and domain-specific natural languages \cite{fowler_parsons_2010}.
DSVLs are often different types of diagrams and may often have an accompanying programmatic DSL \cite{dsvl}.
In this report the term DSL is going to refer to the more generally accepted notion of a DSL as being a programming language or other executable specification language, except for when explicitly referring to DSVLs or other kinds DSLs.


%``DSLs trade generality for expressiveness in a limited domain.'' \cite[p.~317]{mernik_heering_sloane_2005}.


\subsubsection{Embedded Domain-Specific Languages}\label{sec:edsl}
%%\cite[p.123]{fowler_parsons_2010} qoute this.

Implementing DSLs from scratch takes a lot of time and effort, and it is therefore common to implement them within another (general-purpose) language \cite{buildingdsel}, called a host language \cite{techniquesforedsls}. This kind of DSL is referred to as an embedded DSL (EDSL). These are relatively easy to both implement and extend \cite{[needed]}. A secondary advantage of this that the host language can exist as a sub-language to the DSL to provide general-purpose capabilities if need be \cite{annotatedbibl}.

Functional languages such as Haskell are popular as host languages for their capabilities of defining custom operators, using higher-order types, and extensive overloading \cite{techniquesforedsls}.

%\todo[inline,color=other]{WiP}
In figure \ref{code:gadts_dsl}, an example of a small EDSL is presented. When the syntax of the DSL is encoded into the algebraic data type as constructors the DSL is called a deeply embedded DSL, as opposed to a shallow embedding as is shown in figure \ref{code:shallow}. Shallow embeddings do not have the same clear distinction between syntax and semantics. The usage of this DSL always instantly evaluates, since the syntax and semantics are embedded in the same structure. In contrast, a deep embedding---where the constructors encodes the structure (syntax) and the semantic evaluation separately---will determine the result of that structure at some later point in time.

The native differentiation of syntax and semantics in deeply embedded DSLs maps quite well onto differentiating technical frameworks' notation and the evaluation of that structure. 

There are many language features of Haskell that simplify the implementation of a embedded DSLs. %, which in conjunction with language extensions---i.e. extensions that teach/augment the compiler how to type infer or adds language constructs [Ref]---offers a strong framework for dealing with DSLs. 
The most noteworthy extension for this project is Generalised Algebraic Datatypes, GADTs, which adds Haskell syntax that facilitates declaring algebraic data types much like one would declare a Haskell instance or class.

\begin{figure}[ht]
    \begin{minipage}{.499\linewidth}
        \centering
        \inputminted[autogobble]{haskell}{code/data.hs}
        \caption{Deep DSL with GADTs language extension.} 
        \label{code:gadts_dsl}
    \end{minipage}
    \begin{minipage}{.499\linewidth}
        \inputminted[autogobble]{haskell}{code/data_no_GATDs.hs}
        \caption{Deep addition DSL without language extensions.} 
        \label{code:vanilla_dsl}                    
    \end{minipage}\par\medskip %% Adds small space between rows.
    \begin{minipage}{.499\linewidth}
        \inputminted{haskell}{code/class.hs}  
        \caption{Declaring a Haskell class.} 
        \label{code:class}
    \end{minipage}
    \begin{minipage}{.499\linewidth}
        \inputminted[autogobble]{haskell}{code/shallow_dsl.hs}  
        \caption{Shallow DSL.} 
        \label{code:shallow}
    \end{minipage}
\end{figure}

\noindent
In the code in figure \ref{code:gadts_dsl} a small DSL over addition is implemented, where GADTs allows class-like syntax when declaring a new algebraic data type.
Figure \ref{code:vanilla_dsl} shows a version of figure \ref{code:gadts_dsl} without the GADTs extension. These three snippets precisely describe the same behaviour in Haskell, but GADTs offers a type declaration syntax closer to that of type theory---making these types easily readable. 

%...
%``DSL development is hard, requiring both domain and language development expertise. Few people have both.'' \cite[p.320]{mernik_heering_sloane_2005}
%...
\subsection{Control Theory}
Control theory deals with dynamic continuous systems and their control. Control theory as a subject is taught in various programmes at Chalmers. The computer science students of Chalmers are taught control theory in their third year in the course \course{Control theory}{Reglerteknik}{ERE103}.

Although Control theory as a subject is relatively new, the techniques therein have been used for a long time. The Romans used control theory to control the water levels in their aqueducts and it was later used to control the velocity of windmills \cite{ControlTheoryHistory}. Today control theory is utilised in many technical advanced systems like cell phone networks and fighter jets.

A system can be controlled be one of two basic loops. Open-loop or closed-loop. The closed-loops, or feedback-loop utilise the output of the process to control the system. The open-loop works without the use of feedback.

\newpage
\subfile{Didactics}
\newpage
\subfile{process}
\newpage
\subfile{Produkt}
\newpage
\subfile{discussion}
\newpage
\subfile{conclusion}

\newpage
\printbibliography
\newpage
\appendix
%\listoffigures
%\listoftables
\section{Learning material}\label{app_learning_material}
Here follows a snapshot as of \textbf{date} of the learning material that was developed during this project. Please note that this is a work still in progress and to see an up to date version please visit \url{https://github.com/simonhagglund/DATX02-dsl}.


ladda ner aktuell version och byt rad (jobbitgt att ladda hela varje gång) skriv även ned datum här och i sec:learning material.
\includepdf[page={1,2,3}]{produkt.pdf}



\end{document}
