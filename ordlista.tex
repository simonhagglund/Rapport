

\newglossaryentry{DSVL}{
        name=DSVL,
        description={See Domain-Specific Visual Language}
}

\newglossaryentry{DSVL2}{
        name=Domain-Specific Visual Language,
        description={A DSL that is not programmatic, but visual. They are often based on different types of diagrams}
}

\newglossaryentry{DSL}
{
        name=DSL,
        description={See Domain-Specific Language},
}

\newglossaryentry{DSL2}
{
        name={Domain-Specific Language},
        description={A programming language created and tailored for a specific narrow domain}
}

\newglossaryentry{EDSL}
{
        name=EDSL,
        description={Embedded domain-specific language, a DSL implemented within another, general purpose, language}
}


\newglossaryentry{DSLsofMath}
{
        name=DSLsofMath, 
        description={Short for ``Domain-specific languages of Mathematics,'' a project and related BSc-level course at Chalmers}
}

\newglossaryentry{didactic}
{
        name=Didactics,
        text=didactic,
        description={The science of teaching}
}
\newglossaryentry{GADTs}
{
        name=GADTs, 
        description={See Generalised Algebraic Datatypes}
}
\newglossaryentry{Generalised Algebraic Data Types}
{
        name=Generalised Algebraic Datatypes, 
        description={A non-standard extension of Haskell, enabling more advanced type behaviour}
}
\newglossaryentry{Haskell}{
        name=Haskell,
        description={A functional programming language}
}

\newglossaryentry{ZPD}
{
        name=ZPD, 
        description={Zone of Proximal Development, 
        the collection of tasks the student can do only with guidance\todo{does this work?}
        %refer to the difference between what a student can achieve alone compared to when aided by a mentor
        }
}

\newglossaryentry{Linear Time-Invariant}
{
        name=Linear Time-Invariant,
        description={Systems where the output does not depend on when the input was applied and where the output is linearly related to the output}
}

\newglossaryentry{LTI}
{
        name=LTI,
        description={See \gls{Linear Time-Invariant}}
}

\newglossaryentry{ERE103}
{
        name=ERE103,
        description={Course code for the course ``Control theory'' at Chalmers}
}

\newglossaryentry{controltheory}{
        name={Control theory}, 
        text={control theory}, 
        description={An area of study which deals with the control of continuously operating dynamical systems in engineered processes and machines}
}

%\newglossaryentry{minutes}{
%        name={Minutes},
%        description={Chalmers' favoured method of course evaluation; based on feedback gathered from students at the conclusion of each course}
%}

\newglossaryentry{TIF085}{
    name={TIF085},
    description={Course code for ``Physics for Engineers''}
}

\newglossaryentry{SSY080}{
    name={SSY080},
    description={Course code for ``Transforms, Signals and Systems''}
}

\newglossaryentry{DAT326}{
    name={DAT326},
    description={Course code for ``Domain-Specific Languages of Mathematics''}
}

\iffalse
\newglossaryentry{SQL}{
    name = {SQL},
    description= {Structured Query Language, a domain-specific language used for database queries}
}
\fi

\iffalse
\newglossaryentry{HTML}{
    name = {HTML},
    description= {Hypertext Markup Language, a domain-specific language used to describe web pages 
    %web-browsers
    }
}
\fi


\iffalse
DSL 


tss


physics with dsls


control theory (reglerteknik

Dslofmath

gadt

Learning material


LTI

Haskell

laplace

time and frequency domain

EDSL

didaktiktermer, får nog fixas efter den delen är klar
\fi

\glsdisablehyper