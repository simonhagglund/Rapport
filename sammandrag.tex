\iffalse
\todo[inline,color=other]{Vanlig utsträckning för denna text är uppmot en sida. Det här är en sammanfattning av de viktigaste delarna i vår text. Essentiellt att få med vilka ideer som ledde till detta projekt. Här behöver vi inte förklara vad saker som domänspecifika språk, reglerteknik och haskell är för nåt. det kommer i själva rapporten. Vi presenterar vårat projekt och resultaten(:/). Vi berättar vilka utvecklingsområden som finns och vilka projekt som skulle kunna bygga vidare på detta.}
\noindent

Checklista sammandrag (från writing.chalmers.se):
\begin{itemize}
\item Inledning: en kortkort bakgrund som också visar vikten av den här studien.
\item studiens frågeställning eller syfte (problemformulering om sådan finns)
\item metod – vad det är för typ av studie, hur har resultatet tagits fram? 
\item resultat – vad studien visar
\item slutsats – vad ni kommit fram till.  
\end{itemize}
\fi

%\todo[inline]{\textbf{wip/draft}: förslag på innehållslig uppbyggnad, behöver skrivas bättre}
%\todo[inline, color=other]{writing.chalmers.se säger att ett sammandrag ska innehålla: inledning, frågeställning/syfte, metod, resultat och slutsats.}

% inledning
%Att röra sig mellan discipliner med olika tillvägagångssätt att lära ut kan ibland vara svårt
\noindent
Olika discipliner använder olika tillvägagångssätt för att undervisa, vilket kan skapa svårigheter för de studenter som rör sig mellan dessa discipliner. Tidigare forskning has föreslagit att \begin{modtext}lärtekniker \end{modtext} från datavetenskap skall appliceras för att förbättra lärandet hos datavetenskapsstudenter inom matematiska ämnen. Bland annat inkluderar dessa lärotekniker domänspecifika språk, typteori, och \begin{modtext}tydlig \end{modtext} uppdelning mellan syntax och semantik. \begin{modtext}Projektet ''Domain-Specific Languages of Mathematics'' (''DSLsofMath'') har använt dessa tekniker och resulterat \end{modtext} både i en kurs på Chalmers universitet och tillhörande kursmaterial. I kursen implementeras domänspecifika språk för matematiska koncept, och matematiken analyseras från ett datavetenskapligt perspektiv. \begin{modtext}Studenter \end{modtext}som avklarat kursen har påvisat ökade chanser att även klara \begin{modtext}två senare kurser som underkänt många av deras klasskamrater\end{modtext}.

% vad vi försöker göra
Denna rapport presenterar det supplementära läromaterialet ''Domain-specific languages of Control Theory - A supplementary learning material for ERE103''. Det är speciellt utformat för Chalmers' Reglerteknikkurs ERE103, \begin{newtext}vilken är en av tidigare nämnda kurser som underkänt många\end{newtext}, och tar inspiration från DSLsofMath. För att avgöra vad som skulle ingå i läromaterialet användes återkoppling från både kursansvarig och elever. Tanken var att fånga de avsnitt som elever haft svårast att greppa, samt de avsnitt som effektivast kan förklaras via domänspecifika språk. Läromaterialet beskriver implementationen av ett flertal domänspecifika språk för reglerteknik med hjälp av programmeringsspråket Haskell, och diskuterar även hur dessa fungerar. Läromaterialet är uppdelat mellan avsnitt som behandlar förkunskaper till kursen; integraler, komplexa tal och Laplace transformen, samt avsnitt som hanterar mer centrala koncept inom reglerteknik, såsom LTI-system, överföringsfunktioner och Nyquistkriteriet. Det används även en del didaktisk teori för att utöka effekten av läromaterialet.

% Framtida projekt
Framtida projekt skulle kunna baseras på att fortsätta utöka läromaterialet och de domänspecifika språken som skapats. De skulle även kunna evaluera projektet empiriskt, eller utveckla ytterligare läromaterial för andra ämnen.

% Nyckelord
\vfill
\noindent Nyckelord: Domänspecifika språk, DSL, läromaterial, reglerteknik, Haskell